\documentclass[12pt]{article}
\usepackage{amsmath,amssymb,amsthm}
\usepackage{fullpage}
\usepackage{graphicx}
\usepackage{hyperref}
\usepackage{url}
\usepackage{booktabs}

\theoremstyle{definition}
\newtheorem{thm}{Theorem}[section]
\newtheorem{lem}[thm]{Lemma}
\newtheorem{corr}[thm]{Corrolary}
\newtheorem{defn}{Definition}[section]
\newtheorem{conj}{Conjecture}[section]
\newtheorem{prob}{Open problem}[section]
\newcommand{\floor}[1]{\left\lfloor #1 \right\rfloor}
\newcommand{\ceil}[1]{\left\lceil #1 \right\rceil}
\newcommand{\bigC}[0]{\mathcal{C}}
\begin{document}
\emergencystretch 3em
\title{Is detecting more cliques harder?}

\author{Josh Burdick ({\tt josh.t.burdick@gmail.com})}

\maketitle

\begin{abstract}



Shannon's counting argument \cite{shannon_synthesis_1949} shows that many functions are hard to compute.
However, it is nonconstructive, and so doesn't explicitly state a
hard-to-compute function. Attempts have been made to apply counting arguments to other problems
such as CLIQUE \cite{buggyclique}, without success.
Here, we define a random walk on a $d$-regular graph whose vertices are
sets of cliques. We state a (huge) integer programming
problem which appears to bound the circuit complexity of CLIQUE. We show solutions
of that problem (for {\em tiny} cases -- $n=7, k=3$).

\end{abstract}

\newpage

\tableofcontents

\vspace{5mm}

\section{Graphs with more vertices require more gates, {\em on average}}



\bibliography{references}
\bibliographystyle{unsrt}

\end{document}
